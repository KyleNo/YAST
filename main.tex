\documentclass[oneside]{book}
\usepackage[margin=1.2in]{geometry}
\usepackage[cache=false]{minted}
\usepackage{tikz}
\usepackage[utf8]{inputenc}
\usepackage{xcolor}
\usepackage{amssymb}
\usepackage{ifthen}
\usepackage[most, minted]{tcolorbox}

\usepackage{a4wide}
\usepackage{array}
\newcolumntype{C}{>{\centering\arraybackslash}X}
\renewcommand{\arraystretch}{1.2}
\usepackage{booktabs}
\usepackage{tabularx}
\usepackage{caption}
\usepackage{array}
\usepackage{float}
\usepackage{natbib}
\usepackage{graphicx}
\usepackage{blindtext}
\usepackage{tocloft}

\usepackage[hidelinks]{hyperref}
\usepackage{vhistory}

% \immediate\write18{which pygmentize > pygmentize-path.txt}
% \pagecolor{gray}
% \color{white}
\definecolor{bgdark}{HTML}{2E3440}  % Nord-ish dark gray-blue
\definecolor{fgtext}{HTML}{ECEFF4} % Near-white
\definecolor{arrowcolor}{HTML}{88C0D0}
\definecolor{darkline}{HTML}{4C566A}

\definecolor{codebg}{HTML}{2E3440}
\definecolor{codelinebg}{HTML}{3B4252}
\definecolor{codelinefg}{HTML}{88C0D0}
\definecolor{codefg}{HTML}{ECEFF4}
\definecolor{arrowdark}{HTML}{88C0D0}
\definecolor{arrowlight}{HTML}{333333}
\definecolor{bglight}{HTML}{FAFAFA}
\definecolor{bgdark}{HTML}{242424}
\definecolor{linenobgdark}{HTML}{2A2D2E}
\definecolor{borderdark}{HTML}{2A2D2E}



\newboolean{darkmodeon}
\setboolean{darkmodeon}{true}   % <-- set to false for light mode

\ifthenelse{\boolean{darkmodeon}}
% Dark Mode
{\newtcblisting{msccode}{
  minted language=msc2,
  minted style=msc_dark,
  minted options={
      fontsize=\normalsize,
      linenos,
      numbersep=1mm,
      breaksymbolleft=\textcolor{arrowdark}{\tiny$\hookrightarrow$},
      breaklines=true,
  },% <-- put other minted options inside the brackets
  overlay={%
      \begin{tcbclipinterior}
          \fill[linenobgdark] (frame.south west) rectangle ([xshift=5mm]frame.north west);
      \end{tcbclipinterior}
  },
  colback=bgdark,
  colframe=borderdark,
  before skip=5pt plus 2pt,
  breakable,
  enhanced,% <-- put other tcolorbox options here
  listing only
  }}
% Light Mode
{\newtcblisting{msccode}{
  minted language=msc2,
  minted style=msc_light,
  minted options={
      fontsize=\normalsize,
      linenos,
      numbersep=1mm,
      breaksymbolleft=\textcolor{arrowlight}{\tiny$\hookrightarrow$},
      breaklines=true,
  },% <-- put other minted options inside the brackets
  overlay={%
      \begin{tcbclipinterior}
          \fill[gray!25] (frame.south west) rectangle ([xshift=5mm]frame.north west);
      \end{tcbclipinterior}
  },
  colback=bglight,
  colframe=black!70,
  before skip=5pt plus 2pt,
  breakable,
  enhanced,% <-- put other tcolorbox options here
  listing only
  }}

\ifthenelse{\boolean{darkmodeon}}
  % Dark Mode
  {\newtcblisting{nmscode}{
    minted language=nms,
    minted style=msc_dark,
    minted options={
        fontsize=\normalsize,
        linenos,
        numbersep=1mm,
        breaksymbolleft=\textcolor{arrowdark}{\tiny$\hookrightarrow$},
        breaklines=true,
    },% <-- put other minted options inside the brackets
    overlay={%
        \begin{tcbclipinterior}
            \fill[linenobgdark] (frame.south west) rectangle ([xshift=5mm]frame.north west);
        \end{tcbclipinterior}
    },
    colback=bgdark,
    colframe=borderdark,
    before skip=5pt plus 2pt,
    breakable,
    enhanced,% <-- put other tcolorbox options here
    listing only
    }}
  % Light Mode
  {\newtcblisting{nmscode}{
    minted language=nms,
    minted style=msc_light,
    minted options={
        fontsize=\normalsize,
        linenos,
        numbersep=1mm,
        breaksymbolleft=\textcolor{arrowlight}{\tiny$\hookrightarrow$},
        breaklines=true,
    },% <-- put other minted options inside the brackets
    overlay={%
        \begin{tcbclipinterior}
            \fill[gray!25] (frame.south west) rectangle ([xshift=5mm]frame.north west);
        \end{tcbclipinterior}
    },
    colback=bglight,
    colframe=black!70,
    before skip=5pt plus 2pt,
    breakable,
    enhanced,% <-- put other tcolorbox options here
    listing only
}}

% \renewcommand{\theFancyVerbLine}{\ttfamily
%   \textcolor[rgb]{0,0,0}{\small{\arabic{FancyVerbLine}}}}

\ifthenelse{\boolean{darkmodeon}}
{\renewcommand{\theFancyVerbLine}{\ttfamily
\textcolor[HTML]{707880}{\small{\arabic{FancyVerbLine}}}}}
{\renewcommand{\theFancyVerbLine}{\ttfamily
\textcolor[rgb]{0,0,0}{\small{\arabic{FancyVerbLine}}}}}

%%% Preface environment
% ===== Define abstract environment =====
\newcommand{\prefacename}{Preface}
\newenvironment{preface}{
    \vspace*{\stretch{2}}
    {\noindent \bfseries \Huge \prefacename}
    \begin{center}
        \phantomsection \addcontentsline{toc}{chapter}{\prefacename} % enable this if you want to put the preface in the table of contents
        \thispagestyle{plain}
    \end{center}%
}
{\vspace*{\stretch{5}}}

%%%%% END OF PREAMBLE %%%%%

\begin{document}

\pagestyle{empty}

\title{Yet Another Scripting Tetorial \\
        \large For Minr Scripts 2.4.7}
\author{Tetration}
\date{\vhCurrentDate \\Version \vhCurrentVersion}
\maketitle

% \shipout\null

\frontmatter
\pagenumbering{roman} 

\begin{preface}
\section{Why is This a Thing?}
\blindtext

\section{Who is This For?}
\blindtext

\section{What Will This Cover?}
\blindtext
\end{preface}


\tableofcontents


\mainmatter
\pagestyle{plain}

\chapter{Environment}
\blindtext

\section{Welcome to the Test Server!}
\blindtext

\section{Hastebin}
\blindtext

\section{Visual Studio Code}
\blindtext

\section{Source Control}
\blindtext

\section{Tips}
\blindtext

\section{Summary}
\blindtext

\chapter{The Basics}

\section{Triggers}
\blindtext

\section{What is a Script?}
\blindtext

\section{Running Commands}
\blindtext

\section{Printing to the Player}
\blindtext

\section{Delay}
\blindtext

\section{Values and Types}
\blindtext

\section{Expressions}
\blindtext

\section{Variables}
\blindtext

\section{Summary}
\blindtext

\section{Exercises}
\blindtext

\chapter{Core Topics}
\blindtext

\section{Lists}

\section{Control Flow}
\blindtext

\section{Namespaces}
\blindtext

\section{Variable Qualifiers}
\blindtext

\section{Functions}
\blindtext

\section{Meta Operators}
\blindtext

\section{Summary}
\blindtext

\section{Exercises}
\blindtext

\chapter{Advanced Topics}
\blindtext

\section{User-Defined Types}
\blindtext

\section{Schematics}
\blindtext

\section{Chat Scripts}
\blindtext

\section{More about Lists and Strings}
\blindtext

\section{Built-in Namespaces}
\blindtext

\section{Built-in Types}
\blindtext

\section{Complexity}
\blindtext

\section{Best Practices}
\blindtext

\section{Learn to Learn}
\blindtext

\chapter{Special Topics}
\blindtext

\section{Block Displays}
\blindtext

\section{Animating Entities}
\blindtext

\section{Abstraction}
\blindtext

\section{Books}
\blindtext

\section{Scoreboards}
\blindtext

\section{Boss Bars}
\blindtext

\section{Importing Scripts with Scripts}
\blindtext

\section{Execute}
\blindtext

\section{Resource Packs}
\blindtext

\section{Debug Tools}
\blindtext

\section{Recursion}
\blindtext

\section{Attributes}
\blindtext

\section{Summary}
\blindtext

\section{Exercises}
\blindtext

\chapter{Recipes}

\section{Stateful Puzzles}
\blindtext

\section{Search Maps}
\blindtext

\section{Managing Heads}
\blindtext


% \begin{msccode}
% # Long prng::uuidToSeed(String uuid, Long modulo)

% @using prng
% @player Hello? test

% @define Long seed = 0l

% @define Boolean boo3 = true

% @var myNamespace::myObj.myMethod(0, 2, 3)

% @var myNamespace::myFunction("test")

% @var seed = normalize(seed).toLowerCase().length()

% @define String hex = "0123456789abcdef"
% @define String test_string = "seed is {{seed}}"
% @define Int j = 0

% @var uuid = uuid.replace("-", "").toLowerCase()
% @define Int l = uuid.length()-9

% @for Int i in list::range(0,l)
%     @player loop iteration {{i}}
%     @var j = l-i-1
%     @var seed = (seed + hex.indexOf(uuid.substring(j, j+1)) * pow(16l, i)) % modulo
% @done

% @define Box box = Box()

% @return seed
% \end{msccode}

% \begin{nmscode}
% @namespace myNamespace
% 	# myFunc documentation
% 	Double myFunc(Player player, Item item)
% 	myVoidFunc()
% 	# myVar documentation
% 	Double myVar

% 	# myClass documentation
% 	@class MyClass
% 		# constructor
% 		MyClass(Double value)
% 		# another constructor
% 		MyClass(Double value1, Double value2)
% 		# field
% 		Double x
% 		# getter
% 		Double getX()
% 		# setter
% 		setX(Double newValue)
% 		myNamespace::MyClass getMyClass()
% 	@endclass
% @endnamespace
% \end{nmscode}

\begin{versionhistory}
    \vhEntry{1.0}{\today}{Tetration}{First Publication}
\end{versionhistory}

\end{document}
